\chapter{Introdução} \label{chap:intro}

\section*{}

Nuvens computacionais são um serviço emergente na indústria tecnológica, pois permitem computação a um custo reduzido. A própria tecnologia é apenas a adaptação de outras tecnologias existentes para responder a um crescimento de utilização de serviços na \textit{internet}. Este serviço disponibilizado ao público por empresas como a Google, Amazon ou Microsoft permite a expansão do poder computacional aos utilizadores conforme as suas necessidades, eliminando um investimento inicial e possibilitando o crescimento do serviço. Grandes organizações, como o caso do CERN e NSA, utilizam redes computacionais internas para disponibilizar o poder computacional que as suas tarefas diárias necessitam, não limitando cada colaborador ao poder do seu computador pessoal, permitindo a aproveitação dos recursos organizacionais para a conclusão de tarefas mais rapidamente. 

A presente dissertação apresentará um método para melhorar o serviço das nuvens computacionais através dum escalonamento de tarefas mais consciente sobre o processo de virtualização e das interferências que o mesmo provoca na realização de tarefas nos nós computacionais.

% O primeiro capítulo da dissertação deve servir para apresentar o
% enquadramento e a moti\-va\-ção do trabalho e para identificar e
% definir os problemas que a dissertação aborda.
% Deve resumir as metodologias utilizadas no trabalho e termina
% apresentando um breve resumo de cada um dos capítulos
% posteriores.

% Este documento ilustra o formato a usar em dissertações na \Feup, não
% servindo de exemplo sobre os conteúdos a usar.
% São dados exemplos de margens, cabeçalhos, títulos, paginação, estilos
% de índices, etc. 
% São ainda dados exemplos de formatação de citações, figuras e tabelas,
% equações, referências cruzadas, lista de referências e índices.

% Uma recolha de normas existentes sobre este assunto pode ser
% encontrada em~\cite{kn:Mat93}. 

% \begin{quote}
%   ``Like the Abstract, the Introduction should be written to engage the
%   interest of the reader. It should also give the reader an idea of
%   how the dissertation is structured, and in doing so, define the
%   thread of the contents.''~\cite[chap.\ Introduction]{kn:Tha01} 
% \end{quote}

% Neste primeiro capítulo ilustra-se a utilização de citações e de
% referências biblio\-grá\-fi\-cas.
% Para além de dar um exemplo de utilização de uma citação, a citação
% anterior, introduz uma referência que pode ser consultada, entre
% muitas outras referências bibliográficas
% interessantes~\cite{kn:Tha01,kn:PP05}. 

% \section{Contexto} \label{sec:context}

% Esta secção descreve a área em que o trabalho se insere, podendo
% referir um eventual projeto de que faz parte e apresentar uma breve
% descrição da empresa onde o trabalho decorreu.

% Lorem ipsum~\cite{kn:Lip08} dolor sit amet, consectetuer adipiscing
% elit. 


\section{Motivação e Objetivos} \label{sec:goals}

A otimização da utilização dos recursos disponíveis permite melhorias na qualidade do serviço (QoS) fornecido aos utilizadores, através de serviços mais rápidos e responsivos, e menor gastos por parte dos fornecedores, através de reduções energéticas devido a menores períodos de utilização e cumprimento dos acordos da qualidade do serviço (SLA), influenciando positivamente todas as partes interessadas. É, portanto, uma área de interesse estudar otimizações possíveis que contribuam para a eficiência do serviço. Uma parte principal do funcionamento de uma nuvem é o escalonador, responsável por atribuir as tarefas aos nós disponíveis. Esta atribuição é feita em linha, aquando da chegada de uma tarefa, o escalonador deverá avaliar os vários nós da rede computacional e despachar a tarefa para o nó que possua a melhor avaliação.
Esta avaliação tem em conta as características de uma tarefa, através da especificação do pedido ou por execução da mesma temporariamente e medir os recursos exigidos, e as capacidades e quantidade de trabalho de cada nó. Técnicas de escalonamento atuais não tem em conta a interação entre tarefas, um dos principais culpados nos atrasos na nuvem, devido às tecnologias de virtualização e arquitetura do \textit{hardware} usados. Quando duas tarefas com utilizações de recursos semelhantes, por exemplo ambas tem bastante acesso a disco para recolha e escrita de dados com pouco tratamento dos mesmos a nível do processador, são atribuídas ao mesmo nó, que terá espaço de disco para ambas, as máquinas virtuais irão interferir entre si no acesso a recursos que não podem ser divididos para cada virtualização devido à sua arquitetura, neste caso, o barramento de dados. 

Esta dissertação tem como objetivo desenvolver um escalonador que utilize métricas de interferência para poder reduzir o impacto das mesmas na nuvem. Para isso será estudada a interferência entre vários tipos de tarefas, em bibliografia existente, e utilizado um simulador de uma nuvem computacional para avaliar o desempenho do escalonador. Para efeitos de avaliação de resultados será utilizado um escalonador de código aberto, utilizado atualmente na indústria, o \textit{Filter Scheduler} do OpenStack \cite{gong2012nova}. Para simulação será utilizado o SimGrid e a arquitetura de rede será idêntica à utilizada no CERN, podendo para simulação de tarefas ser utilizado os dados de utilização da sua nuvem ou criadas novas tarefas de uso científico geral, como especificadas em \cite{mehta2013pegasus}.



% Apresenta a motivação e enumera os objetivos do trabalho terminando
% com um resumo das metodologias para a prossecução dos objetivos.


\section{Estrutura da Dissertação} \label{sec:struct}

Os capítulos seguintes deste relatório encontram-se da forma seguinte. O capítulo 2 apresenta o estado da arte na área de computação em nuvem, sendo específico na área de escalonamento e de interferências na virtualização. O capítulo 3 apresenta os dados sobre interferência recolhidos e explica o impacto dos mesmos na rede computacional. O capítulo 4 descreve detalhes de implementação que serão utilizados para a realização do trabalho e os métodos utilizados para avaliação de resultados. O capítulo 5 apresentará as conclusões retiradas da preparação de dissertação e o plano de trabalho a seguir.

% Para além da introdução, esta dissertação contém mais x capítulos.
% No capítulo~\ref{chap:sota}, é descrito o estado da arte e são
% apresentados trabalhos relacionados. 
% %\todoline{Complete the document structure.}
% No capítulo~\ref{chap:chap3}, ipsum dolor sit amet, consectetuer
% adipiscing elit.
% No capítulo~\ref{chap:chap4} praesent sit amet sem. 
% No capítulo~\ref{chap:concl}  posuere, ante non tristique
% consectetuer, dui elit scelerisque augue, eu vehicula nibh nisi ac
% est. 
