\chapter{Implementação}\label{chap:implementacao}

\section*{}

% Este capítulo pode ser dedicado à apresentação de detalhes de nível
% mais baixo relacionados com o enquadramento e implementação das
% soluções preconizadas no capítulo anterior.
% Note-se no entanto que detalhes desnecessários à compreensão do
% trabalho devem ser remetidos para anexos.

% Dependendo do volume, a avaliação do trabalho pode ser incluída neste
% capítulo ou pode constituir um capítulo separado.

\section{Algoritmo de Escalonamento}

%\todofigure{Inserir uma figura sobre o Map/Reduce}


%\todoline{Escrever sobre o map/reduce}

%\todolines{A short entry in the list of todos}{A very long todonote
%  that certainly will fill more than a single line in the list of
%  todos. Just to make sure let's add some more text.} 


\section{Validação de Resultados}

%\todoref{Citar Map/reduce}


% \begin{lstlisting}[float,language=Java, label=src:mapreduce, caption=Example map and reduce functions for word counting]
% map(String key, String value): 
% // key: document name 
% // value: document contents 
% for each word w in value:
% EmitIntermediate(w, "1");

% reduce(String key, Iterator values):
% // key: a word 
% // values: a list of counts 
% int result = 0;
% for each v in values: 
% result += ParseInt(v);

% Emit(AsString(result))
% \end{lstlisting}



\section{Resumo ou Conclusões}


